\documentclass{report}

\usepackage[utf8]{inputenc}
\usepackage[italian]{babel}
\usepackage{import}
\usepackage{todonotes}
\usepackage{color}
\usepackage{rotating}
\usepackage{hyperref}
\usepackage{url}
\usepackage{pdfpages}
\usepackage{siunitx}
\usepackage[italiano]{algorithm2e}
\usepackage{pdflscape}
\usepackage{subfig}

\usepackage{listings}
\usepackage{xcolor}

\usepackage[signatures,swapnames,sans]{frontespizio}

\definecolor{codegreen}{rgb}{0,0.6,0}
\definecolor{codegray}{rgb}{0.5,0.5,0.5}
\definecolor{codepurple}{rgb}{0.58,0,0.82}
\definecolor{backcolour}{rgb}{0.95,0.95,0.92}

\lstdefinestyle{mystyle}{
	backgroundcolor=\color{backcolour},   
	commentstyle=\color{codegreen},
	keywordstyle=\color{magenta},
	numberstyle=\tiny\color{codegray},
	stringstyle=\color{codepurple},
	basicstyle=\ttfamily\footnotesize,
	breakatwhitespace=false,         
	breaklines=true,                 
	captionpos=b,                    
	keepspaces=true,                 
	numbers=left,                    
	numbersep=5pt,                  
	showspaces=false,                
	showstringspaces=false,
	showtabs=false,                  
	tabsize=2
}

\definecolor{eclipseStrings}{RGB}{42,0.0,255}
\definecolor{eclipseKeywords}{RGB}{127,0,85}
\colorlet{numb}{magenta!60!black}

\lstset{style=mystyle}

\usepackage{geometry}
\geometry{portrait, margin=3cm}

\newcommand{\Fig}[0]{Fig.}
\newcommand{\Alg}[0]{Alg.}

\begin{document}
	
	\begin{frontespizio}
		\Margini{3cm}{3cm}{3cm}{3cm}
		\Universita{Bergamo}
		\Logo[43.332mm]{unibg-mark}
		\Divisione{Scuola di Ingegneria}
		\Corso[Laurea Magistrale]{Ingegneria Informatica}
		\Titolo{Progetto corso di Programmazione avanzata}
		\Sottotitolo{C++ e Haskell}
		\Punteggiatura{}
		\NRelatore{Prof.}{}
		\Relatore{Angelo Gargantini}
		\Candidato[1057926]{Matteo Verzeroli}
		\Annoaccademico{2021--2022}
		\begin{Preambolo*}
			\usepackage[italian]{babel}
			\usepackage[T1]{fontenc}
			\usepackage[utf8]{inputenc}
			\usepackage{microtype}
			\usepackage{lmodern}
			\graphicspath{{img/}}
			
			\renewcommand{\frontinstitutionfont}{\fontsize{14}{17}\bfseries\scshape}
			\renewcommand{\fronttitlefont}{\fontsize{17}{21}\bfseries\scshape}
			\renewcommand{\frontfootfont}{\fontsize{12}{14}\bfseries\scshape}
		\end{Preambolo*}
	\end{frontespizio}	
	
	\tableofcontents
	
	\chapter{Progetto C++}
	
	\import{./TextFiles}{Progetto C++.tex}
	
	\chapter{Progetto Haskell}
	
	\import{./TextFiles}{Progetto Haskell.tex}
	
\end{document}


\begin{comment}
	le classi "repository" sono singleton
	utilizzo stl con map
	utilizzo shared ptr nelle map
	utilizzo di auto negli iteratori stampa utenti
	
	
	ma nel patter singleton devo fare la delete della instanza?
\end{comment}