\section{Introduzione}
Il programma realizzato permette di gestire utenti e interventi di una squadra della Protezione Civile, utilizzando una semplice interfaccia grafica realizzata tramite il framework Qt.
In particolare, gli utenti sono suddivisi in base al loro ruolo:
\begin{itemize}
	\item volontario;
	\item caposquadra;
	\item amministratore.
\end{itemize}
Ogni utente, in base al proprio ruolo, avrà accesso a particolari funzionalità quali aggiungere o cancellare un utente nel sistema, visualizzare gli utenti della propria squadra, visualizzare e pianificare interventi. A differenza degli amministratori, il volontario e il caposquadra appartengono a una squadra, caratterizzata da un nome e da un'area di intervento. 
Per accedere all'applicazione è necessario effettuare un login, inserendo uno username, rappresentato dal proprio numero di matricola e una password scelta al momento della registrazione dell'utente.
L'interfaccia grafica è stata realizzata attraverso gli oggetti QtWidget che permettono di semplificare la realizzazione della GUI tramite l'editor Qt Creator.



\section{Funzionamento}



\section{Diagramma delle classi}
